% !TeX spellcheck = en_GB

\documentclass{amsart}
\usepackage[foot]{amsaddr} % put addresses on first page
\usepackage[british]{babel}
\usepackage{geometry}
\usepackage{graphicx,psfrag,epsf}
\usepackage{enumerate}
\usepackage{enumitem}
\usepackage{amsfonts}
\usepackage{mathtools}
\usepackage{amssymb}
\usepackage{amsmath}
\allowdisplaybreaks
\usepackage{longtable}
\usepackage{bigints}
%\usepackage{siunitx}
\usepackage{amsthm}
\usepackage{soul}
\usepackage{tikz}
\usepackage{color}
\usepackage{easyReview} % for \comment, \alert, etc.

\newcommand{\addition}[1]{#1}

\usepackage{hyperref}
\usepackage[capitalise]{cleveref}
\newtheorem{theorem}{Theorem}[section]
\newtheorem{corollary}{Corollary}[theorem]
\newtheorem{lemma}[theorem]{Lemma}
%

\usepackage[numbers]{natbib}
\usepackage{url} % not crucial - just used below for the URL 
\usepackage{doi}

\newcommand{\x}{\boldsymbol{X}}
\renewcommand{\b}{\hat{\beta}}
\newcommand{\reals}{\mathbb{R}}
\newcommand{\normal}{\mathcal{N}}
\newcommand{\lexp}{\underline{\text{E}}}
\newcommand{\uexp}{\overline{\text{E}}}


\keywords{high dimensional data; variable selection; Bayesian analysis; imprecise probability}

\begin{document}

\title{Robust Bayesian Analysis of Causal Inference Problems}
\author{Tathagata Basu$^1$}
\email{tathagatabasumaths@gmail.com}
\author{Matthias C.~M.~Troffaes$^2$}
\email{matthias.troffaes@durham.ac.uk}
\author{Jochen Einbeck$^{2,3}$}
\email{jochen.einbeck@durham.ac.uk}
\address{$^1$Civil and Environmental Engineering, University of Strathclyde}
\address{$^2$Department of Mathematical Sciences, Durham University}
\address{$^3$Durham Research Methods Centre}

\begin{abstract}
Causal inference using observational data is an important aspect in
many fields such as epidemiology, social science, economics, etc. In
particular, our goal is to find the treatment effect on the subjects
along with the causal links between the variables and the outcome.
However, estimation for such problems are extremely 
difficult as the treatment effects may vary from subject 
to subject and modelling the underlying heterogeneity explicitly makes the 
problem practically unsolvable. Another issue we often face is the 
dimensionality of the problem and we need to find a subset of 
explanatory variables to initiate the treatment. However, currently variable selection methods
tend to maximise the predictive performance of the outcome model only. This can be problematic in the case of limited information. 
As the consequence of mistreatment can be harmful. 
So, in this paper, we suggest a general framework with
robust Bayesian analysis which accounts for abstention in deciding
an explanatory variable in the high dimensional regression model. 
To achieve that, we consider a set of spike and slab priors 
through prior elicitation to obtain robust estimates for
both the treatment and outcome model. We are specifically interested 
in the sensitivity of the treatment effect in the high dimensional causal inference
as well as the identifying the confounder variables by means of variable selection. However, indicator
based confounder selection can be deceptive in some cases. Especially, 
when the predictor is strongly associated with either the treatment or 
the outcome. This increases the posterior expectation of the selection
indicators. To avoid that we apply a post-hoc selection scheme
which successfully remove negligible non-zero effects from the model
attaining a smaller set of confounders. Finally, we illustrate
our result using synthetic \alert{and real} dataset.

\end{abstract}

\maketitle

\section{Introduction}\label{sec:intro}

In causal inference, we are interested in estimating the causal
effect of independent variables on a dependent variable. Ideally,
randomised trials are the most efficient way to perform this task.
However, this is not very practical for several reasons; ethical 
concerns, design cost, population size, to name a few. This
leaves us with observational studies which are usually obtained
by means collecting data though surveys or record keeping. But this
can be problematic in the presence of confounders. That is when
the variables are associated with both the treatment and the outcome.
In such cases, we need to be extra cautious as otherwise it will
lead to unwanted bias in the treatment effect estimator \cite{rosenbaum83}.
Several works have been done in order to tackle the presence of
confounder variables. One such work in the topic was 
by \citet{Robins1986ANA} where the author used  a graphical
approach for the identification of the causal parameters.
\citet{rosenbaum1985} suggested the use of a link model to estimate
the propensity scores for all individuals. Later on several other
methods have been proposed based on propensity score matching.
A brief review on such methods can be found in \cite{winship99, stuart10}.

Bayesian approach in causal effect estimation has become a popular
topic in recent days. However, it has been a topic of interest for about 50 
years and one of the earlier works on this can be
found in \cite{rubin1978}. Lately with the notion of high dimensional problems, Bayesian methodologies have become
more appealing. \citet{Crainiceanu2008} proposed a bi-level 
Bayesian model averaging based method for estimating the causal 
effect. \citet{wang2015} suggested BAC (or, Bayesian adjustment for
confounding) where they use an informative prior obtained from
the treatment model and apply them on the outcome model for
estimating causal effect. Several other methods were also
proposed to tackle confounders from the point of view of Bayesian
variable selection such as: \citet{Zigler2014}, \citet{Hahn2018} etc.

In this paper we take inspiration from the approach of \citet{koch2020}, where
they proposed a bi-level spike and slab prior for causal effect 
estimation. They considered a data-driven adaptive approach to
propose their prior which reduce the variance of the causal estimate. 
In our approach, we are interested in the decision theoretic notion of 
variable selection and its effect in causal estimation. We suggest a 
sensitivity analysis based approach where instead of using a single prior, 
we consider a set of priors \cite{BERGER1990303}. This is particularly 
interesting as in many cases, causal effect estimation can be performed 
through a meta analysis and hence robust Bayesian analysis can be beneficial
\cite{raices_cruz22} under severe uncertainty. Moreover, for some problems 
we have to rely on very limited data to perform our Bayesian analysis and 
choosing a data-driven prior can lead to overfitting. Instead we can 
benefit from expert opinion and elicit prior based on empirical evidence.
This also allows us to construct the problem of confounder selection 
in a framework where abstention has a positive gain. To propose our 
framework, we consider a set of continuous spike and slab priors 
\cite{ishwaran2005} for confounder selection and construct a Bayesian 
group LASSO \cite{xu2015} type problem. To perform the prior sensitivity 
analyses, we consider a set of beta priors on the covariate selection 
probability of the spike and slab priors. We use the posteriors of these
covariate selection probability for identifying the confounders. Finally, 
we consider a post-hoc coefficient adjustment method \citet{hahn2015}
to recover sparse estimates. This post hoc treatment also helps us to 
reduce the bias of treatment effect estimation. 

The rest of the paper is organised as follows. In \cref{sec:causal}
we give a formal description of causal estimation problem in the
context of linear regression. \cref{sec:bayes} is focused on the
Bayesian analysis of causal inference problems, followed by the
motivation of a robust Bayesian analysis along with our proposed decision 
theoretic framework for confounder (variable) selection. In \cref{sec:sim}, 
we provide our result of simulation studies under different scenarios 
and data analysis with \alert{real data} in \cref{sec:data:analysis}. Finally 
we discuss our findings and conclude this paper in \cref{sec:conc}.

\section{Causal Estimation}\label{sec:causal}

Let an observational study give us the outcomes $Y=(Y_1$, \dots, $Y_n)$ along with 
corresponding treatment indicators $T=(T_1$, \dots, $T_n)$. Then the
treatment 
effect in the population is given by the expectation of the difference
in outcome between the treatment and controls. 
\begin{align}
\delta = \mathbb{E}(Y\mid T =1) - E(Y\mid T=0).
\end{align}
Similarly, individual causal
effect of the treatment $T_i$ on outcome $Y_i$ is given by:
\begin{align}
\delta_i \coloneqq (Y_i\mid T_i=1) - (Y_i\mid T_i=0).
\end{align}
That is, the difference between the outcome when $i$-th subject receives
the treatment and when $i$-th subject remain as a control. 

In theory, both of these quantities exist.
However, we can not observe $(Y_i\mid T_i=1)$ and $(Y_i\mid T_i=0)$
simultaneously for the $i$-th individual. Instead, we can estimate
average causal effect of the treatment $T$ by calculating the averaged
outcome of all the subjects those received the treatment and
all the subjects those remained as control.
\begin{align}
\hat{\delta} \coloneqq 
\frac{\sum_{i=1}^n Y_i\cdot\mathbb{I}(T_i=1) - 
	\sum_{i=1}^n Y_i\cdot\mathbb{I}(T_i=0)}{n}.
\end{align}
However, this relies on an important assumption that the treatment effect
on the $i$-th subject given that they received the treatment is
equal to the treatment effect when they remain as the control
\cite{winship99}.

\subsection{Regression Model}
Regression methods are widely used in causal effect estimation. The
main idea behind these regression methods is to remove the
correlation between the treatment indicator and the error term
\cite{winship99,HECKMAN1985}. To do so, we rely on $p$ different observed quantities
or predictors denoted by $X\coloneqq$ $[X_1$, \dots, $X_n]^T$. Now, let
$\beta \coloneqq (\beta_1$, \dots, $\beta_p)$ denotes the vector of regression
coefficients. Then we can define a linear model for the outcome
so that
\begin{equation}
    Y_i = \beta_{T} T_i + \beta_0 + X_i\beta + \epsilon_i
\end{equation}
where $\epsilon_i\sim \mathcal{N}(0, \sigma^2)$. Clearly, when
the underlying true outcome model is linear,
\begin{equation}
    \delta = \beta_{T}\quad\text{and hence } 
    \hat{\delta} - \delta = \hat{\beta}_T - \beta_T.
\end{equation}
\begin{figure}
	\centering
	\begin{tikzpicture}[params/.style={circle, draw=green!60, fill=green!5, very thick, minimum size=7mm}]
	\node (1) at (0,0) {Treatment ($T$)};
	\node (2) [right = of 1] {Outcome ($Y$)};
	\node (4) [below = of 1] {Predictors ($X$)};
	
	\path (1) edge[->]  (2);
	\path (1) edge[<-]  (4);
	\path (2) edge[<-] (4);
	\end{tikzpicture}
	\caption{Confounding in causal models.}
	\label{fig:confounding}
\end{figure}

In the presence of confounders we also need to consider the
association between the treatment indicators and the predictors.
\citet{koch2020} suggested the use of a probit link function to 
construct the regression model. This way, we can
specify the conditional probability that subject $i$ receives the treatment through a linear model. 
That is, for another vector of regression coefficients 
$\gamma\coloneqq(\gamma_1, \cdots, \gamma_p)$ we define
\begin{align}
    P(T_i=1\mid X_i) \coloneqq \Phi(\gamma_0+X_i\gamma)
\end{align}
where $\Phi(\cdot)$ denotes the cumulative distribution function
of a standard normal distribution.

This allows us to define intermediate latent variables as suggested
by \citet{albert93}
\begin{equation}
	T_i^* = \gamma_0 + X_i\gamma +u_i
\end{equation}
where, $u_i\sim\mathcal{N}(0,1)$. Therefore, $T_i=1$ if $T_i^*>0$ and
$T_i=0$ if $T_i^*\le0$. 

Now, following the approach of \citet{koch2018}, we define an adjusted
output vector $W\coloneqq(Y, T^*)^T$ and corresponding $2n\times(2p+3)$ dimensional design matrix
\begin{align}
    Z &=
    \begin{bmatrix}
     X_O & 0 \\
     0 & X_T
    \end{bmatrix},
\end{align}
where, $X_O = [T, 1_n, X]$ and $X_T = [1_n, X]$. Then, assuming a
Gaussian error term, we have the following likelihood distribution
\begin{align}
W\mid Z, \alpha, \beta, \gamma, \sigma^2 \sim\normal\left(Z\nu, \Sigma\right)\label{eq:like:group},
\end{align}
where $\nu = (\beta_T, \beta_0, \beta, \gamma_0, \gamma)^T$ and
\begin{align}
\Sigma &=
\begin{bmatrix}
\sigma^2{I}_n & 0 \\
0 & {I}_n
\end{bmatrix}.
\end{align}


\section{Bayesian Causal Estimation}\label{sec:bayes}

The likelihood formation given by \cref{eq:like:group} gives us
a foundation for Bayesian group LASSO 
\cite{xu2015} type model and look into the posterior selection
probability associated with the $j$-th predictor. There are several
ways to construct spike and slab priors which achieve 
variable selection. In our case, we consider a continuous type
\cite{ishwaran2005} prior for faster posterior
computation.


\subsection{Hierarchical model}

Let, $\pi_j$ denote the prior probability that the $j$-th
predictor is not associated with either the outcome or the 
treatment. That is, 
\begin{equation}
	\pi_j = P\left((\beta_j,\gamma_j)=(0,0)\right).
\end{equation}
Then we can define a spike and slab group LASSO so that:
for $1\le j\le p$,
\begin{align}
(\beta_j,\gamma_j)^T \mid \pi_{j}, \sigma^2 &\sim 
(1-\pi_{j})\normal\left( (0,0)^T, 
\tau_1^2\begin{bmatrix}
\sigma^2 & 0 \\
0 & 1
\end{bmatrix}\right)
+ \pi_{j} \normal\left(0, 
\tau_0^2\begin{bmatrix}
\sigma^2 & 0 \\
0 & 1
\end{bmatrix}\right)\\
%\beta_T, \beta_0\mid \sigma^2 &\sim \normal(0, \sigma^2)\\
%\gamma_0 &\sim \normal(0,1)\\
\sigma^2&\sim \text{InvGamma}(a, b)\\
\pi_{j} &\sim\text{Beta}\left(sq_j, s(1-q_j)\right).
\end{align}

We fix sufficiently small $\tau^2_0$
$(1\gg\tau_0^2>0)$ so that  $(\beta_j, \gamma_j) = (0,0)$ has its probability mass 
concentrated around zero. Therefore, this represents the spike component of our prior specification. 
To construct the slab component, we consider $\tau_1^2$ to be large so that $\tau_1^2\ge 1$. This allows the prior for $(\beta_j, \gamma_j)\not=(0,0)$ to be flat. Inverse-Gamma is a natural choice
for the variance of the Gaussian noise because of conjugacy. To
make the prior flat, we consider $1\gg a, b >0$.

As indicated earlier, $\pi_j$ is used as the rejection probability
of the $j$-th predictor and we use a beta prior to specify 
these rejection probabilities where  $q_j$ represents our prior expectation of the rejection probability ($\pi_j$) and `$s$' acts as 
a concentration parameter.
For the intercept terms of the outcome model and the causal effect, 
we consider a sufficiently flat prior so that 
$\beta_0. \beta_T\sim \normal(0,\sigma^2)$. Similarly, for the
intercept term in the treatment model, we consider 
$\gamma_0\sim \normal(0,1)$. 

In \cref{fig:regress}, we show a probabilistic graphical representation
of our hierarchical model. In the figure, grey circular nodes represent the
prior hyper-parameters which will be used for sensitivity analysis
of the model. The transparent circular nodes are used to denote
the modelling parameters which are our quantities of interest. 
The observed quantities are denoted with transparent rectangular
nodes. We also use a grey rectangular node to denote the intermediate
latent variable $T^*$. We use directed edges to denote the
relationship between different nodes. However, we use a dashed
edge between $X$ and $T$ as we they are related through the latent
variable $T^*$. This dashed edge also establish the notion of
confounding as shown in \cref{fig:confounding}.

\begin{figure}
	\centering
	\begin{tikzpicture}[params/.style={circle, draw=black!60, very thick, minimum size=7mm},
	hyper/.style={circle, draw=black!60, fill=black!20, thick, minimum size=7mm},
	post/.style={circle, draw=black!60, fill=green!20, thick, minimum size=7mm},
	latent/.style={rectangle, draw=black!60, fill=black!10, dashed, minimum size=7mm},
	data/.style={rectangle, draw=black!60, thick, minimum size=8mm}]
	\node[params] (1) at (0,0) {$\pi$};
	\node[data] (2) at (3,0) {$X$};
	\node[data] (3) at (6,0) {$T$};
	\node[params] (4) at (1.5,1.5) {$\gamma$};
	\node[latent] (5) at (4.5,1.5) {$T^*$};
	\node[params] (6) at (1.5,-1.5) {$\beta$};
	\node[data] (7) at (3.5,-1.5) {$Y$};
	\node[params] (8) at (0,-3) {$\sigma^2$};
	\node[params] (9) at (3,-3) {$\beta_0$};
	\node[params] (10) at (6,-1.5) {$\beta_T$};
	\node[hyper] (11) at (-1.5,-.8) {$s$};
	\node[hyper] (12) at (-1.5,.8) {$q$};
	\node[hyper] (13) at (-1.5,-2.2) {$a$};
	\node[hyper] (14) at (-1.5,-3.8) {$b$};
	\draw[black, dashed] (0.75,-3.7) rectangle (7,2.1);
	
	\path (1) edge[->]  (6);
	\path (1) edge[->]  (4);
	\path (8) edge[->]  (6);
	\path (6) edge[->]  (7);
	\path (2) edge[->]  (7);
	\path (2) edge[->]  (5);
	\path (5) edge[<-] (4);
	\path (5) edge[->] (3);
	\path (3) edge[->]  (7);
	\path (9) edge[->]  (7);
	\path (10) edge[->]  (7);
	\path (2) edge[dashed][->] (3);
	\path (8) edge[bend right = 60][->]  (10);
	\path (8) edge[->]  (9);
	\path (11) edge[->]  (1);
	\path (12) edge[->]  (1);
	\path (13) edge[->]  (8);
	\path (14) edge[->]  (8);
	
	\end{tikzpicture}
	\caption{Probabilistic graphical representation for causal inference with Bayesian hierarchical model.}
	\label{fig:regress}
\end{figure}

\subsection{Robust Bayesian Analysis}
The hierarchical model presented above is fairly common in
Bayesian analysis and performs well when we have sufficient data to begin with. However, especially
in the case of causal inference having sufficient data may not be practical. Moreover, since there’s
an aspect of intervention in causal effect estimator, we also need to be cautious around the side
effects of a treatment.
Therefore, we are interested in having a more robust framework for selecting a variable. So
that, once we are initiating our treatment, we can decide whether we require more data to conduct
our analysis. For instance, we can assign a cost for false positives and false negatives and also for
conducting more tests. This way we can have a more generalised utility based framework with
which we can incorporate the notion of indeterminacy.
We perform our
robust Bayesian analysis on $q\coloneqq(q_1$, \dots, $q_p)\in\mathcal{P}$, where
\begin{equation}
	\mathcal{P} \coloneqq \mathcal{P}_1\times\cdots\times\mathcal{P}_p\subseteq \left(0, 1\right)^{p}.
\end{equation}

\subsection{Variable selection and coefficient adjustment}
For the co-variate selection, we look into the posterior expectation of $\pi_j$. 
We consider the $j$-th predictor to be removed from both the
treatment and outcome model, if
\begin{align}
    \lexp (\pi_j\mid W)\coloneqq \inf_{q_j\in \mathcal{P}_j} \text{E}(\pi_j\mid W) > 1/2.
\end{align}

For the rest of the variables, some of them will be present
in the model as confounders and some will only be associated with
either the treatment. Let $\mathcal{S}$ denote the set of
predictors such that,
\begin{equation}
    \mathcal{S}\coloneqq
    \left\{j : \lexp(\pi_j\mid W) < 1/2\right\}.
\end{equation}
That is the set that contains all the variables which are not
removed from both treatment and outcome model. Now, for
each fixed value of $q$,
let $\hat{\beta}_{\mathcal{S}}(q_{\mathcal{S}})$ be the posterior means of the regression coefficients of the outcome model with respect to
the predictors that belong to $\mathcal{S}$. Similarly,
$\hat{\gamma}_{\mathcal{S}}(q_{\mathcal{S}})$ be the posterior means of the regression
coefficients for the treatment effects. Since, we use a continuous type
selection prior, these regression coefficients are non-zero in nature.
Therefore, to adjust the sparsity, we apply the 
``decoupled shrinkage and selection'' method proposed by \citet{hahn2015}. To do so, we solve the following adaptive LASSO-type \cite{Zou2006}
problems

\begin{align}
    \hat{\beta}^*_{\mathcal{S}}(q) &= 
    \arg\min_{\beta_{\mathcal{S}}} \frac{1}{n}\|X_{\mathcal{S}}\hat{\beta}_{\mathcal{S}}(q)
    - X_{\mathcal{S}} \beta_{\mathcal{S}}\|_2^2 + \lambda\sum_{j\in\mathcal{S}} 
    \frac{|\beta_j|}{|\hat{\beta}_j(q_j)|}
\end{align}
and
\begin{align}
    \hat{\gamma}^*_{\mathcal{S}}(q) &= 
    \arg\min_{\gamma_{\mathcal{S}}} \frac{1}{n}\|X_{\mathcal{S}}\hat{\gamma}_{\mathcal{S}}(q)
    - X_{\mathcal{S}} \gamma_{\mathcal{S}}\|_2^2 + \lambda\sum_{j\in\mathcal{S}} 
    \frac{|\gamma_j|}{|\hat{\gamma}_j(q_j)|}
\end{align}
where $q_{j}\in \mathcal{P}_{j}$ for all $j\in\mathcal{S}$.

\iftrue
\subsection{Causal effect adjustment} The DSS method explained
above give us adjusted coefficient estimates for the treatment
and outcome model. However, as result the causal effect estimate
remains the same and modelling with such adjusted sparse effect 
will contribute to the prediction error. Therefore, we need to 
adjust the causal effect estimate as well. We do that in three steps
similar to double residual regression procedure as suggested by
\citet{robinson1988}.

First we compute the predicted values of $Y$ with respect to
the adjusted regression coefficients so that,
\begin{equation}
\hat{Y}(q) 
= \hat{\beta}_0(q) + X_{\mathcal{S}}\hat{\beta}^*_{\mathcal{S}}(q)
\end{equation}
and the values of $T$ with respect to the adjusted
regression coefficients so that
\begin{equation}
\hat{T}(q) 
= \Phi\left(\hat{\gamma}_0(q) + X_{\mathcal{S}}\hat{\gamma}^*_{\mathcal{S}}(q)\right).
\end{equation}
Then we estimate the adjusted causal effect using the following linear
model
\begin{equation}
\left(Y-\hat{Y}(q)\right) = \left(T - \hat{T}(q)\right)\beta_{T} + \epsilon^*_i.
\end{equation}
That is 
\begin{equation}
\beta_{T}^* = \left(\left(T - \hat{T}(q)\right)^T\left(T - \hat{T}(q)\right)\right)^{-1}\left(T - \hat{T}(q)\right)^T\left(Y-\hat{Y}(q)\right)
\end{equation}

\fi


\section{Simulation Studies}\label{sec:sim}

\section{Data Analysis}\label{sec:data:analysis}

\section{Conclusion}\label{sec:conc}

\bibliographystyle{agsm}
\bibliography{basu22}
\end{document}
